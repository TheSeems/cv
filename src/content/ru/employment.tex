\begin{rubric}{История работы}
\entry*[Май 2023 -- $\cdots\cdot$]%
	\textbf{Яндекс,} Middle+ Software Engineer
        \par \textit{Junior (май 2023) $\rightarrow$ Middle (май 2024) $\rightarrow$ Middle+ (август 2025)}
        \par Внутренняя платформа для дата-инженеров: деплой и эксплуатация Flink jobs/кластеров.
        \par Технологии: Flink, ZooKeeper, Spring Boot, PostgreSQL, jOOQ, Liquibase.
        \par\smallskip
        \textbf{Достижения:}
        \vspace{-0.5em}
        \begin{itemize} \itemsep -2pt
        \item Сократил время запуска Flink jobs на \textbf{70\%} (компиляция JobGraph внутри сервиса);
        \item Построил CI/CD для кластеров и jobs; внедрил систему по сбору бизнес-метрик для джобов;
        \item Улучшил алертинг, снизил шум; поддерживал коннекторы (YDB, ClickHouse, YT);
        \item Наставлял стажёра до выхода на full-time.
        \end{itemize}
%
% Blank lines result in extra space!
%
\entry*[Декабрь 2021 -- Май 2023]%
	\textbf{Сбер,} Junior Software Engineer
        \par Продукты поддержки наличного денежного обращения (прогнозирование инкассаций). Мигрировал сервисы во внутреннее облако (OpenStack), разрабатывал интеграционную логику. Технологии: Git, Atlassian stack (Jira, Confluence, Bitbucket), Java, Spring.
%
\entry*[Август 2021 -- Декабрь 2021]%
	\textbf{VBC,} Intern Java Developer
        \par Работал с микросервисной архитектурой (Netflix stack): реализовывал сложную бизнес-логику и оптимизировал взаимодействия с PostgreSQL. Разработал гибкий сервис ценообразования кредитов, учитывающий историю компании/физического лица, банковские правила, лимиты и требования.

\end{rubric}